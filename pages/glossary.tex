% refer to https://en.wikibooks.org/wiki/LaTeX/Glossary for acronyms and glossary entries

\newacronym{vra}{VR}{virtual reality}

\newacronym{cga}{CG}{computer graphics}

\newacronym{hmda}{HMD}{head mounted display}

\newacronym{imua}{IMU}{inertial measurment unit}

\newacronym{ika}{IK}{\gls{ikgl}}

\newacronym{emsa}{EMS}{electrical muscle stimulation}

\newacronym{pha}{PH}{\gls{phgl}}

\newacronym{soea}{SoE}{\gls{soegl}}

\newacronym{nrpa}{NRP}{\gls{nrpgl}}

\newacronym{ara}{AR}{augmented reality}

\newacronym{apia}{API}{application programming interface}


\newglossaryentry{immersiongl}
{
    name = immersion,
    description = {In video games, immersion refers to ''the degree of involvement with a game'' \autocite[p. ~2]{gameImmersion}}
}

\newglossaryentry{kinesthesiagl}
{
    name = kinesthesia,
    description = {Also referred to as proprioception, \enquote{the sensation of body position and movement}\autocite{proprioception}}
}

\newglossaryentry{controllergl}
{
    name = controller,
    description = {in the context of \gls{vra} and gaming refers to the input device}
}

\newglossaryentry{ikgl}
{
    name = {inverse kinematics},
    description = {Inverse kinematics (IK) makes it possible, just from knowing the target position of, for example, a hand attached to an arm with multiple joints, to calculate the required joint angles to place the hand at its target position. This is used in 3D animation to create realistic movement of game characters \autocite[p. ~3f]{bodyTrackingVR}}
}

\newglossaryentry{phgl}
{
    name = {passive haptics},
    description = {see \autoref{section:passiveHaptics}}
}

\newglossaryentry{soegl}
{
    name = {sense of embodiment},
    description = {see \autoref{chapter:VirtualEmbodiment}}
}

\newglossaryentry{nrpgl}
{
    name = Neurorobotics Plaftorm,
    description = {see \autoref{section:NeuroroboticsPlatform}}
}


%TODO remove v
\newacronym[shortplural={D$_{\text{dye}}$}, longplural={donor dye, ex. Alexa 488}]{ddye}{D$_{\text{dye}}$}{donor dye, ex. Alexa 488}

\newacronym[description={\glslink{r0}{F\"{o}rster        distance}}]{R0}{$R_{0}$}{F\"{o}rster distance}

\newglossaryentry{r0}
{
  name=\glslink{R0}{\ensuremath{R_{0}}},
  text=F\"{o}rster distance,
  description={F\"{o}rster distance, where 50\% ...}, 
  sort=R
}

\newglossaryentry{kdeac}
{
  name=\glslink{R0}{\ensuremath{k_{DEAC}}},
  text=$k_{DEAC}$, 
  description={is the rate of deactivation from ... and emission)}, 
  sort=k
}

\newacronym[shortplural={TUM}, longplural={Technical University of Munich}]{tuma}{TUM}
{Technical University of Munich}

\newglossaryentry{tum}
{
  name = TUM,
  description = {A university in Germany, Bavaria, in the city of Munich},
  plural = TUM
}

\newglossaryentry{computer}
{
  name=computer,
  description={is a programmable machine that receives input,
               stores and manipulates data, and provides
               output in a useful format}
}