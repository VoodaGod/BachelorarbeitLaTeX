% !TeX root = ../main.tex

\chapter{Conclusion}

In conclusion it can be said that using the feedback mechanisms for discrepancies at the feet, outlined and evaluated in this thesis, have a positive impact on the quality of embodiment in the \gls{nrpa}. However there is still much work to be done in this field. This thesis focused on methods easily implementable with software and hardware that is widely available to anyone with access to a \gls{vra} setup. In \autoref{chapter:RelatedWork}, some other approaches were discussed, which did not have this limitation, however most of them are fairly inflexible and difficult to set up. It is to be hoped that more research into this field will lead to more generally applicable commercially available hardware to supplement non-intrusive feedback mechanisms such as those proposed in this thesis.


\section{Future Work}

In order to find out which feedback mechanisms should be implemented in any project, a more in-depth user study should be carried out. It should not only focus on the embodiment of feet, but the entire body. Many potential combinations of effects should be evaluated.
\newline
Ways of adding haptic feedback to the \textit{Vive Trackers} should be investigated, as haptic feedback is generally agreed to be very effective and immersive.
\newline
Integrating hardware such as \textit{Impacto} (see \autoref{section:impacto}) using \gls{emsa} also is an approach that should be investigated further.


\section{Final Words}

As the \textit{Neuroroboticts Unity3D Client} matures, I hope some of the feedback mechanisms implemented during the course of this thesis will prove useful. Adapting the solution to only contain the wanted mechanisms should be trivial.
\newline
The user study was very bare-bones, due to the focus on discrepancies at the feet. It was hard to find ways to encourage the user to do things he should not be doing, such as step into objects, for long enough that any feedback mechanisms would apply before the user stepped out again. Another problem turned out to be that the test subjects had differing understandings of the questions, requiring verbal rephrasing and discussion to make sure that the question had its intended effect.
\newline
Due to the time constraints of a Bachelor's Thesis it also was not possible to do a more in-depth user study to evaluate sensible combinations of the implemented effects.
\newline

All in all, I am happy with the way this thesis turned out, as almost all implemented feedback mechanisms worked as intended and were received well.