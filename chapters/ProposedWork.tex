% !TeX root = ../main.tex

\chapter{Proposed Work}\label{chapter:ProposedWork}

In the course of this thesis, feedback mechanisms to alert the user of discrepancies between his tracked body, the 
\textit{local} avatar, and his physically simulated body, the \textit{remote} avatar, will be implemented into the \textit{Neurorobotics Unity3D Client} (see \autoref{section:NeuroroboticsUnity3DClient}). The aim is to encourage the user to prevent, and resolve, such discrepancies, to increase his \gls{soea} (see \autoref{chapter:VirtualEmbodiment}).
\newline
Although the focus of this thesis is on the addition of feet, feedback mechanisms for hands and head will also be implemented. These mechanisms will be discussed further in \autoref{chapter:Implementation}, and will include visual, auditory and haptic feedback.
\newline
Due to the \textit{Neurorobotics Unity3D Client} already having a seperate local and remote avatar implemented, the \textit{simulated surface constraints technique} mentioned in \autoref{section:YouShallNotPass} is already present. The user can place his hand inside an object, but the remote, physically simulated avatar's hand will not follow into the object.
\newline
This will be used as the baseline experience to which the newly added feedback mechanisms will be compared in a user study and evaluation, discussed in \autoref{chapter:Evaluation}.